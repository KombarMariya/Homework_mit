\documentclass{article}
\usepackage[T2A]{fontenc}
\usepackage[utf8]{inputenc} 
\usepackage{amsthm}
\usepackage{amsmath}
\usepackage{amssymb}
\usepackage{amsfonts}
\usepackage{mathrsfs}
\usepackage[12pt]{extsizes}
\usepackage{fancyvrb}
\usepackage{indentfirst}
\usepackage[
  left=2cm, right=2cm, top=2cm, bottom=2cm, headsep=0.2cm, footskip=0.6cm, bindingoffset=0cm
]{geometry}
\usepackage[english,russian]{babel}


\begin{document}
\section*{Вариант 20}
Модель, позволяющая получить разгонные и переходные характеристики компрессора, то есть кривые изменения во времени основных параметров "--- плотности и температуры газа при нагнетании каждой ступени компрессора в период нагрузки и после внесения внешнего возмущающего воздействия:

\begin{equation}
   \frac{V_{x_1} + 0.85 {V_{\text{\cyr{\textit{Н}}}1}}}{V_{\text{\cyr{П}}2} n} \frac{d}{dt} \rho_1(t) + \lambda_2 \rho_1(t) = \lambda_1 \frac{V_{\text{\cyr{П}}1}}{V_{\text{\cyr{П}}2}} \rho_0,
\end{equation}

\begin{equation}
   \frac{V_{x_2} + 0.85 {V_{\text{\cyr{\textit{Н}}}2}}}{V_{\text{\cyr{П}}3} n} \frac{d}{dt} \rho_2(t) + \lambda_3 \rho_2(t) = \lambda_2 \frac{V_{\text{\cyr{П}}2}}{V_{\text{\cyr{П}}3}} \rho_1,
\end{equation}

\begin{equation}
   0.85 {V_{\text{\cyr{\textit{Н}}}3}} \frac{d}{dt} \rho_3(t) = \lambda_3 V_{\text{\cyr{П}}3} n \rho_2(t) - 
   \begin{cases}
      0, &  \rho_3(t) < \rho_c, \\  
      \lambda_3 V_{\text{\cyr{П}}3} n \rho_2(t), & \rho_3(t) \ge \rho_c,
    \end{cases}
\end{equation}

где $V_{\text{\cyr{П}}1}$, $V_{\text{\cyr{П}}2}$, $V_{\text{\cyr{П}}3}$ "--- объемы, описываемые поршнями цилиндров; $V_{\text{\cyr{\textit{Н}}}1}$, $V_{\text{\cyr{\textit{Н}}}2}$, $V_{\text{\cyr{\textit{Н}}}3}$ "--- объемы газовых полостей аппаратуры на нагнетании; $V_{x1}$, $V_{x2}$, $V_{x3}$ "--- объемы газовых полостей аппаратуры после газоохладителей; $\lambda_1$, $\lambda_2$, $\lambda_3$ "--- коэффициенты наполнения цилиндров; $n$ "--- частота вращения вала компрессора; $\rho_c$ "--- плотность газа в сети; $\rho_1$, $\rho_2$, $\rho_3$ "--- искомые плотности газа при нагнетании по ступеням.

\end{document}

